\hypertarget{use-case}{%
\part{ユースケース}\label{use-case}}

第1部の基本文法で学んだことを応用し、具体的なユースケースを元に学んでいきます。

\hypertarget{summary}{%
\section*{目次}\label{summary}}

\subsection*{\texorpdfstring{\hyperlink{setup-local-env}{アプリケーション開発の準備}}{アプリケーション開発の準備}}

アプリケーション開発のためにNode.jsとnpmのインストールなどの準備方法を紹介します。

\subsection*{\texorpdfstring{\hyperlink{usecase-ajax}{ユースケース: Ajax通信}}{ユースケース: Ajax通信}}

ウェブブラウザ上でAjax通信をするユースケースとして、GitHubのユーザーIDからプロフィール情報を取得するアプリケーションを作成しながら、非同期処理について紹介します。

\subsection*{\texorpdfstring{\hyperlink{node-cli}{ユースケース: Node.jsでCLIアプリケーション}}{ユースケース: Node.jsでCLIアプリケーション}}

Node.jsでCLI(コマンドラインインターフェース)アプリケーションを開発する例として、MarkdownをHTMLに変換するツールを作成していきます。また、Node.jsやnpmの使い方を紹介します。

\subsection*{\texorpdfstring{\hyperlink{todo-app}{ユースケース: Todoアプリケーション}}{ユースケース: Todoアプリケーション}}

ブラウザで動作するウェブアプリケーションの例としてTodoアプリを作成しながら、モジュールを使ったコード管理について紹介します。
